\begin{graphicspathcontext}{{./chapters/chapter1/imgs/raw/},{./chapters/chapter1/imgs/auto/}}
	
	\resolvepicturename{howto}
	\addfancyboxpicture{howto}{\resolvedfilename}
	
	\resolvepicturename{whatthis}
	\addfancyboxpicture{whatthis}{\resolvedfilename}
	
	\resolvepicturename{strong_typed}
	\addfancyboxpicture{strongly-typed}{\resolvedfilename}
	
	\resolvepicturename{weak_typed}
	\addfancyboxpicture{weakly-typed}{\resolvedfilename}
	
	\resolvepicturename{parser_icon}
	\addfancyboxpicture{parser-icon}{\resolvedfilename}
	
	\resolvepicturename{tokenizer}
	\addfancyboxpicture{scanner-icon}{\resolvedfilename}
	
	\resolvepicturename{assembly}
	\addfancyboxpicture{assembly-icon}{\resolvedfilename}
	
	\resolvepicturename{machine_code}
	\addfancyboxpicture{machine-code-icon}{\resolvedfilename}
	
	\resolvepicturename{data_flow}
	\addfancyboxpicture{data-flow-icon}{\resolvedfilename}
	
	\resolvepicturename{toolkit}
	\addfancyboxpicture{toolkit-icon}{\resolvedfilename}
	
\end{graphicspathcontext}

\newcommand{\code}[1]{\ifmmode\text{\lstinline[style=lststyle-java,basicstyle=\normalsize]{#1}}\else\lstinline[style=lststyle-java,basicstyle=\normalsize]{#1}\fi\xspace}
\newcommand{\ccode}[1]{\texttt{#1}\xspace}
\let\id\ccode

\lstdefinestyle{lststyle-c}{%
	language=C,
	% choose the background color
	%backgroundcolor=\color{white},
	% the size of the fonts that are used for the code
	basicstyle=\scriptsize,
	% sets if automatic breaks should only happen at whitespace
	breakatwhitespace=false,
	% sets automatic line breaking
	breaklines=true,
	% sets the caption-position to bottom
	captionpos=b,
	% comment style
	commentstyle=\color{CIADgreen},
	% if you want to delete keywords from the given language
	%deletekeywords={...},
	% if you want to add LaTeX within your code
	escapeinside={\%*}{*)},
	% lets you use non-ASCII characters; for 8-bits encodings only, does not work with UTF-8
	%extendedchars=true,
	% adds a frame around the code
	%frame=single,
	% keeps spaces in text, useful for keeping indentation of code (possibly needs columns=flexible)
	keepspaces=true,
	% keyword style
	keywordstyle=\bfseries,
	% the language of the code
	%language=Octave,
	% if you want to add more keywords to the set
	%morekeywords={*,...},
	% where to put the line-numbers; possible values are (none, left, right)
	numbers=none,
	% how far the line-numbers are from the code
	%numbersep=5pt,
	% the style that is used for the line-numbers
	%numberstyle=\tiny\color{CIADblue},
	% if not set, the frame-color may be changed on line-breaks within not-black text (e.g. comments (green here))
	rulecolor=\color{CIADdarkgray},
	% show spaces everywhere adding particular underscores; it overrides 'showstringspaces'
	showspaces=false,
	% underline spaces within strings only
	showstringspaces=true,
	% show tabs within strings adding particular underscores
	showtabs=false,
	% the step between two line-numbers. If it's 1, each line will be numbered
	stepnumber=4,
	% string literal style
	stringstyle=\color{CIADmagenta},
	% sets default tabsize to 2 spaces
	tabsize=2,
	% show the filename of files included with \lstinputlisting; also try caption instead of title
	%title=\lstname
}

\lstdefinestyle{lststyle-java}{%
	language=java,
	% choose the background color
	%backgroundcolor=\color{white},
	% the size of the fonts that are used for the code
	basicstyle=\scriptsize,
	% sets if automatic breaks should only happen at whitespace
	breakatwhitespace=false,
	% sets automatic line breaking
	breaklines=true,
	% sets the caption-position to bottom
	captionpos=b,
	% comment style
	commentstyle=\color{CIADgreen},
	% if you want to delete keywords from the given language
	%deletekeywords={...},
	% if you want to add LaTeX within your code
	escapeinside={\%*}{*)},
	% lets you use non-ASCII characters; for 8-bits encodings only, does not work with UTF-8
	%extendedchars=true,
	% adds a frame around the code
	%frame=single,
	% keeps spaces in text, useful for keeping indentation of code (possibly needs columns=flexible)
	keepspaces=true,
	% keyword style
	keywordstyle=\bfseries,
	% the language of the code
	%language=Octave,
	% if you want to add more keywords to the set
	%morekeywords={*,...},
	% where to put the line-numbers; possible values are (none, left, right)
	numbers=none,
	% how far the line-numbers are from the code
	%numbersep=5pt,
	% the style that is used for the line-numbers
	%numberstyle=\tiny\color{CIADblue},
	% if not set, the frame-color may be changed on line-breaks within not-black text (e.g. comments (green here))
	rulecolor=\color{CIADdarkgray},
	% show spaces everywhere adding particular underscores; it overrides 'showstringspaces'
	showspaces=false,
	% underline spaces within strings only
	showstringspaces=true,
	% show tabs within strings adding particular underscores
	showtabs=false,
	% the step between two line-numbers. If it's 1, each line will be numbered
	stepnumber=4,
	% string literal style
	stringstyle=\color{CIADmagenta},
	% sets default tabsize to 2 spaces
	tabsize=2,
	% show the filename of files included with \lstinputlisting; also try caption instead of title
	%title=\lstname
}

\endinput
